\newpage
\section{Infraestrutura}

\subsection{Espa\c{c}o F\'isico Geral}

O quadro a seguir apresenta a estrutura física m\'inima necessária ao funcionamento do Curso Tecn\'ologo em Sistemas para Internet e dispon\'ivel no campus Guarabira. Os demais quadros apresentam a relação detalhada dos equipamentos para os laboratórios.

\begin{table}[h]
\caption{Estrutura F\'isica do Campus}
\begin{center}
\begin{tabular}{|p{4.5cm}|p{2.2cm}|p{2.2cm}|p{2.2cm}|p{3.0cm}|}
\hline
Tipo de \'Area & Quantidade necess\'ario & Quantidade dispon\'ivel & \'Area  & Hor\'ario de Funcionamento\\
\hline 
\hline
Salas de aula &  6 & 28 (verificar) & (verificar) & diurno/noturno \\
\hline
Audit\'orios/Anfiteatros &  1 & 1 & (verificar) & diurno/noturno \\
\hline
Salas de professores & 1 & 2 (verificar) & (verificar) & diurno/noturno \\
\hline
\'Areas de apoio acad\^emico & 1 & 1 (verificar) & (verificar) & diurno/noturno \\
\hline
\'Areas Administrativas & X & X (verificar) & (verificar) & diurno/noturno \\
\hline
Coveni\^encia/pra\c{c}as & 1 & 1 (verificar) & (verificar) & diurno/noturno \\
\hline
Banheiros & 1 & 4 (verificar) & 4 (verificar) & diurno/noturno \\
\hline
Laborat\'orios & 2 & 2 (verificar) & 4 (verificar) & diurno/noturno \\
\hline
Biblioteca & 1 & 1 (verificar) & 4 (verificar) & diurno/noturno \\
\hline
\end{tabular} 
\end{center}
\label{tab:pl}
\end{table}
%colocar subtopicos

\subsubsection{Infraestrutura de Seguran\c{c}a}

A prevenção de lesões aos trabalhadores requer a introdução de alterações, dos padrões de trabalho, tais como a passagem de horários noturnos para diurnos, o melhoramento das condições de contratação, valorizando a qualidade do serviço em detrimento do preço, e melhorando a relação entre o docente e discente, podem reduzir diretamente o risco de les\~oes.

Os perigos e riscos que os professores enfrentam incluem:

\begin{itemize}
\item Exposição a substâncias perigosas, incluindo agentes biológicos que podem causar asma, alergias, e infecções no sangue;
Ruído e vibração;
\item Escorregamento, tropeções e quedas durante "o trabalho em piso molhado";
\item Acidentes de origem elétrica provocados pelo equipamento de trabalho;
\item Risco de lesões musculoesqueléticas;
\item Trabalho solitário, estresse profissional, violência, e assédio moral (bullying);
\item Ritmos e horários de trabalho irregulares.
\end{itemize}

\subsubsection{Recursos Audiovisuais e Multim\'idia}

No quadro a seguir estão especificados os equipamentos audiovisuais a serem utilizados pelos professores e alunos do curso.

\begin{table}[h]
\caption{Rela\c{c}\~ao de recursos audiovisuais e multim\'idia}
\begin{center}
\begin{tabular}{|p{4.5cm}|p{2.2cm}|p{2.2cm}|p{2.2cm}|p{3.0cm}|}
\hline
Tipo de Equipamento & Quantidade necess\'ario & Observa\c{c}\~oes\\
\hline 
\hline
TV LED 50'' &  6 & 28 (verificar) & (verificar) & diurno/noturno \\
\hline
Audit\'orios/Anfiteatros &  1 & 1 & (verificar) & diurno/noturno \\
\hline
Salas de professores & 1 & 2 (verificar) & (verificar) & diurno/noturno \\
\hline
\'Areas de apoio acad\^emico & 1 & 1 (verificar) & (verificar) & diurno/noturno \\
\hline
\'Areas Administrativas & X & X (verificar) & (verificar) & diurno/noturno \\
\hline
Coveni\^encia/pra\c{c}as & 1 & 1 (verificar) & (verificar) & diurno/noturno \\
\hline
Banheiros & 1 & 4 (verificar) & 4 (verificar) & diurno/noturno \\
\hline
Laborat\'orios & 2 & 2 (verificar) & 4 (verificar) & diurno/noturno \\
\hline
Biblioteca & 1 & 1 (verificar) & 4 (verificar) & diurno/noturno \\
\hline
\end{tabular} 
\end{center}
\label{tab:pl}
\end{table}

Quadro 18- Relação de recursos audiovisuais e multimídia
TIPO DE EQUIPAMENTO
QUANTIDADE
OBSERVAÇÕES
TV LED 50”
32
Localizadas em cada sala de aula
Projetor multimídia
10
Disponíveis para os laboratórios
Quadro Branco
42
Localizados em cada sala e laboratórios
Lousa digital
1
Disponível também para o Curso de Tecnologia em Telemática
Computadores
80
Distribuídos nos laboratórios do curso


\subsection{Espa\c{c}os F\'isicos Utilizados no Desenvolvimento do Curso}

%colocar subtopicos

\subsection{Biblioteca}

%colocar subtopicos

\subsection{Laborat\'orios e Ambientes Espec\'ificos para o Curso}

%colocar subtopicos 


